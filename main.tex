\documentclass[12pt]{beamer}
\usepackage[T2A]{fontenc}
\usepackage[utf8]{inputenc}
\usepackage[russian]{babel}
\usepackage{hyperref}
\usepackage{tikz}
\usepackage{minted}
\title{Научно-исследовательская практика. Flipping Cookie}
\author{Сухов Максим, Гервятович Олег}
\date{1 Июля 2022}
\usetheme{Dresden}
\begin{document}
%\maketitle
\begin{titlepage}
       
\end{titlepage}
\section{Ссылки}
\begin{frame}{Ссылки}
     \href{https://github.com/ogerv/praktika}{Репа с файлами}
\end{frame}
\section{Условие задачи}
\begin{frame}{Условие задачи}
    Дано какое-то значение cookie, которое состоит из двух частей:
    \begin{enumerate}
        \item Инициализирующий вектор (IV) — первые 16 байт.
        \item Зашифрованная строка 
    \end{enumerate}  
    После с помощью некоторой фукнции мы получаем флаг  
\end{frame}

\section{Решение}
\begin{frame}{Решение}
    Мы расшифрованную строку проверяем на наличие “admin=True;”, если реальность такова, то мы получаем флаг.
    \parНо на деле мы видим, что в изначальной строке записано “admin=False;” и мы не можем управлять этим текстом. Но мы можем менять либо шифротекст, либо инициализирующий вектор.
\end{frame}

\begin{frame}{Решение}
    Менять мы будем инициализирущий вектор (IV), далее мы вспоминаем про XOR и его свойства
\end{frame}

\begin{frame}{Решение}
  \begin{enumerate}
        \item У нас есть инициализирующий вектор:
        \par\textbf{IV = fff44b182176a547ed2cf05320cd7ac4}
        \item При ксоре данных, полученных после расшифровки (T) с IV мы получим строку:
        \par\textbf{IV xor T = “admin=False;…”}
        \item И к этому мы хотим прийти:
        \par\textbf{IV xor T = “admin=True;…”}
    \end{enumerate}  
\end{frame}

\begin{frame}{Решение}
Для этого сделаем следующее:
  \begin{enumerate}
        \item Определим данные, получаемые при расшифровке:
        \par\footnotesize{T = IV xor “admin=False;” = 9e9026714f4be326815f956841a917ad}
        \item\normalsize{Изменим IV:}
        \par\footnotesize{IV = T xor “admin=True;” = fff44b182176b754f43aae0925c47ec3}
    \end{enumerate}
    Осталось только отправить серверу шифротекст с этим инициализирующим вектором, чтобы получить флаг
\end{frame}
\section{Исходный код}
\begin{frame}[fragile]
    \frametitle{Source Code}
    \rule{\textwidth}{1pt}
    \scriptsize
    \begin{minted}{python}
##Нужен для запроса на криптохак
import requests
##Нужен для того, чтобы десериализовать ответ с криптохака(куку)
import json
## XOR, тут и так понятно
from Crypto.Util.strxor import strxor

##Получает с криптохака куку и десереализует ее в текст
def get_cookie():
	return requests.get('https://aes.cryptohack.org/...).json()

##Отправляет куку с iv на криптохак и получает флаг
def get_flag(cookie, iv):
	return requests.get('https://aes.cryptohack.org/...).json()
##Кука с криптохака
Cookie=get_cookie()['cookie']
##iv, выдранный из куки
IV=bytes.fromhex(Cookie[:32])
    \end{minted} 
\end{frame}
\begin{frame}[fragile]
    \frametitle{Source Code}
    \rule{\textwidth}{1pt}
    \scriptsize
    \begin{minted}{python}
##потребуются для подмены
FalseAdminBytes=b'admin=False;expi'  
TrueAdminBytes=b'admin=True;expir'

##Для того, чтобы подменить значение admin-а мы "ксорим" iv 
##и значение admin=false, т.к. "ксорить" саму зашифрованную строку нет смысла
XORedIVFalse=strxor(IV,FalseAdminBytes)
##Тут мы получаем модифицированный iv,
##который уже при расшифровке даст нам admin=true
XORedIVTrue=strxor(TrueAdminBytes,XORedIVFalse).hex()

print(get_flag(Cookie[32:],XORedIVTrue)["flag"])
    \end{minted}
\end{frame}
\end{document}
